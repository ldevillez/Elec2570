\makeatletter
%
% better to create a family, but as an example...
\tikzset{flip flop/port labels/.initial={\tiny}}
%
% number of ports
\tikzset{ports/.initial=4}
%
% we need a counter
\newcount\tmp@a

\pgfdeclareshape{dff}{

    % you have to save the relevant parameters as \savedmacro
    \savedmacro\numports{
        \edef\numports{\pgfkeysvalueof{/tikz/ports}}%
    }
    % and \saveddimen
    \saveddimen\pinsdelta{
        % you can't use savedmacros nor savedanchors here (bummer!)
        \edef\numports{\pgfkeysvalueof{/tikz/ports}}%
        \pgfmathsetlength\pgf@x{0.22*3cm/(\numports+1)}%
    }

    % The 'minimum width' and 'minimum height' keys, not the content, determine
    % the size
    \savedanchor\northeast{%
        \pgfmathsetlength\pgf@x{3.5cm}%
        \pgfmathsetlength\pgf@y{5cm}%
        \pgf@x=0.11\pgf@x
        \pgf@y=0.15\pgf@y
    }
    % This is redundant, but makes some things easier:
    \savedanchor\southwest{%
        \pgfmathsetlength\pgf@x{3.5cm}%
        \pgfmathsetlength\pgf@y{5cm}%
        \pgf@x=-0.11\pgf@x
        \pgf@y=-0.15\pgf@y
    }
    % Inherit from rectangle
    \inheritanchorborder[from=rectangle]

    % Define same anchor a normal rectangle has
    \anchor{center}{\pgfpointorigin}
    \anchor{north}{\northeast \pgf@x=0pt}
    \anchor{east}{\northeast \pgf@y=0pt}
    \anchor{south}{\southwest \pgf@x=0pt}
    \anchor{west}{\southwest \pgf@y=0pt}
    \anchor{north east}{\northeast}
    \anchor{north west}{\northeast \pgf@x=-\pgf@x}
    \anchor{south west}{\southwest}
    \anchor{south east}{\southwest \pgf@x=-\pgf@x}
    \anchor{text}{
        \pgfpointorigin
        \advance\pgf@x by -.5\wd\pgfnodeparttextbox%
        \advance\pgf@y by -.5\ht\pgfnodeparttextbox%
        \advance\pgf@y by +.5\dp\pgfnodeparttextbox%
    }

    % Define anchors for signal ports

    \anchor{CLK}{
        \pgf@process{\northeast}%
        \pgf@x=0\pgf@x%
        \pgf@y=-1.1\pgf@y%
    }
	\anchor{C}{
		\pgf@process{\northeast}%
		\pgf@x=0\pgf@x%
		\pgf@y=-\pgf@y%
		\pgfmathsetlength\pgf@xx{1.3ex}
		\advance\pgf@y by -\pgf@xx
	}
    \anchor{D}{
        \pgf@process{\northeast}%
        \pgf@x=-1.5\pgf@x%
        \pgf@y=0\pgf@y%
    }

    \anchor{Q}{
        \pgf@process{\northeast}%
        \pgf@x=1.5\pgf@x%
        \pgf@y=0\pgf@y%
        \pgfmathsetlength\pgf@xx{1.3ex}
		    \advance\pgf@y by \pgf@xx
    }

    \anchor{QN}{
        \pgf@process{\northeast}%
        \pgf@x=1.5\pgf@x%
        \pgf@y=0\pgf@y%
        \pgfmathsetlength\pgf@xx{1.3ex}
		    \advance\pgf@y by -\pgf@xx
    }

    % Draw the rectangle box and the port labels
    \backgroundpath{
        % Rectangle box
        \pgfpathrectanglecorners{\southwest}{\northeast}
        \pgfpathcircle{\pgfpoint{0cm}{-0.79cm}}{0.2ex}

        % Drawing Triangle for clock input
        % upper left x
        \southwest \pgf@xa=\pgf@x \pgf@ya=\pgf@y \pgf@yb=\pgf@y
        \northeast \pgf@xb=\pgf@x
        \pgf@anchor@reg@CLK
        \pgf@xc=\pgf@x \pgf@yc=\pgf@y
        \pgfmathsetlength\pgf@x{1.3ex}
        \advance\pgf@xa by 1.5mm
        \advance\pgf@xb by -1.5mm
        \advance\pgf@yc by \pgf@x
        \pgfpathmoveto{\pgfpoint{\pgf@xa}{\pgf@ya}}
        \pgfpathlineto{\pgfpoint{\pgf@xb}{\pgf@yb}}
        \pgfpathlineto{\pgfpoint{\pgf@xc}{\pgf@yc}}
        \pgfclosepath


        \tikzset{flip flop/port labels} % Use font from this style
        \tikz@textfont



        %Drawing CLK circuit
        \pgf@anchor@reg@south
        \pgf@ya=\pgf@y
        \pgf@anchor@reg@CLK
        \pgf@xa=\pgf@x 
        \pgf@xb=\pgf@x \pgf@yb=\pgf@y
        \pgfmathsetlength\pgf@x{1.8ex}
        \advance\pgf@yb by -\pgf@x
        \pgfmathsetlength\pgf@x{0.8ex}
        \advance\pgf@ya by -\pgf@x
        \pgfpathmoveto{\pgfpoint{\pgf@xa}{\pgf@ya}}
        \pgfpathlineto{\pgfpoint{\pgf@xb}{\pgf@yb}}
        %Draw clock label
        %\pgf@anchor@reg@CLK\pgftext[base,at={\pgfpoint{\pgf@x}{\pgf@y}}]{\raisebox{-2.5ex}{CLK}}

        %Drawing PC circuit
        \pgf@anchor@reg@D
        \pgf@ya=\pgf@y \pgf@yb=\pgf@y \pgf@xa=\pgf@x
        \pgf@anchor@reg@west
        \pgf@xb=\pgf@x
        %\pgfmathsetlength\pgf@x{2.7ex}
        %\advance\pgf@xb by \pgf@x
        \pgfpathmoveto{\pgfpoint{\pgf@xa}{\pgf@ya}}
        \pgfpathlineto{\pgfpoint{\pgf@xb}{\pgf@yb}}
        \pgf@anchor@reg@D\pgftext[base,at={\pgfpoint{\pgf@x+3.5ex}{\pgf@y}}]{\raisebox{-0.7ex}{D}}

        %Drawing PC' circuit
        \pgf@anchor@reg@east
        \pgf@xb=\pgf@x
        \pgf@anchor@reg@northeast
        \pgf@xa=1.5\pgf@x%
        \pgf@ya=0\pgf@y%
        \pgfmathsetlength\pgf@xx{2.6ex}
		    \advance\pgf@ya by \pgf@xx
        \pgf@yb = \pgf@ya
        \pgfpathmoveto{\pgfpoint{\pgf@xa}{\pgf@ya}}
        \pgfpathlineto{\pgfpoint{\pgf@xb}{\pgf@yb}}
        \pgf@anchor@reg@Q\pgftext[base,at={\pgfpoint{\pgf@x-3.5ex}{\pgf@y+2.6ex}}]{\raisebox{-0.7ex}{Q}}

        % QN
        \pgf@anchor@reg@northeast
        \pgf@ya=0\pgf@y%
        \pgfmathsetlength\pgf@xx{2.6ex}
		    \advance\pgf@ya by -\pgf@xx
        \pgf@yb = \pgf@ya

        \pgfpathmoveto{\pgfpoint{\pgf@xa}{\pgf@ya}}
        \pgfpathlineto{\pgfpoint{\pgf@xb}{\pgf@yb}}
        \pgf@anchor@reg@Q\pgftext[base,at={\pgfpoint{\pgf@x-4.3ex}{\pgf@y-2.6ex}}]{\raisebox{-0.7ex}{QN}}
    }

    % create input anchors
    % this touch internal things, so beware...
    % anchors are named pgf@anchor@<name-of-the-shape>@<name of the anchors>
    \pgfutil@g@addto@macro\pgf@sh@s@reg{%
        \tmp@a=\numports\relax
        \pgfmathloop%
        \ifnum\pgfmathcounter>\tmp@a%
        \else%
        % assign the anchor "in \pgfmathcounter" to the macro \reg@port with the number as argument
        \expandafter\xdef\csname pgf@anchor@reg@in \pgfmathcounter\endcsname{%
            \noexpand\reg@port{\pgfmathcounter}% defined below
        }%
        % \typeout{YAY\space\pgfmathcounter}
        \repeatpgfmathloop%
    }

}
%
\def\reg@port#1{%
    % this macro has the function to return the position of the anchor
    % it must use only \savedanchors and \savedmacros
    % the parameter is the number of the anchor (see above)
    \northeast
    \pgf@x=-\pgf@x
    \pgf@ya=\pgf@y
    \pgfmathsetlength{\pgf@y}{\pgf@ya-(#1+0.5)*\pinsdelta}%
}
\makeatother
