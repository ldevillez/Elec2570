

\subsection{Physical implementation}
\textbf{Step 1:} The floorplanning
\begin{itemize}
  \item the core size (dimension or the aspect ratio)
  \item the core to boundary distance.
  \item Core row utilisation
    \subitem High utilization (lower area and possible shorter routing (lower delay - low C)
    \subitem Low utilization (Routing less complicated, use space to do timing optimization, less heat/area)
  \item Row Organization: orientation and spacing
\end{itemize}

\textbf{Step 2:} Placement of the hard macros

\textbf{Step 3:} Placement of the well taps

\textbf{Step 4:} Power rail (width and position \(\rightarrow\) care to IR drop)

\textbf{Step 5:} Clock tree
\begin{itemize}
  \item short path are most prone to hold time violations
  \item Hold time violation cannot be fixed with a reduction of clock frequency
  \item Skew is temporal due to
    \subitem Clock source
    \subitem ambient and circuit noise
    \subitem Supply/ground bouncne
\end{itemize}
The clock tree improves skew and transition but introduce a delay. It degrade
\begin{itemize}
  \item REG2OUT setup closure
  \item IN2REG hold closure
\end{itemize}

\textbf{Step 6:} Routing
\begin{enumerate}
  \item Power nets
  \item Clock nets
  \item Signal nets (timing-driven)
\end{enumerate}
We use buffer and repeater to lower the capacitance to drive (lower \(\tau\)).

Crosstalk and Signal Integrity (SI) \(\Rightarrow\) Lower the cap



% TODO replace this
Need decap cells to reduce load to drive 

\subsection{Timing optimization}
\begin{itemize}
  \item Add/delete buffers
  \item Resize cells
  \item Restructure the netlist
  \item Remap logic
  \item Swap pins
  \item Move instances
\end{itemize}

\subsection{Packaging}
\begin{itemize}
  \item Package parsitics
  \item Wirebond inductance
\end{itemize}
ground/supply bounce at rising edge of the clock (flip-flop toggle) or whene output pad is changing (large C and large transistor). To prevent that
\begin{itemize}
  \item Parallel connection (reduce parasitic inductance), Half IO \(\rightarrow\) power pads
  \item Ground plane
\end{itemize}


\subsection{Technology scaling}

\subsubsection{5V world and happy scaling}
\textbf{Moore's law:} The number of transistor per chip doubles every 1.5-2 years

\textbf{Transistor sclaling:} Dividing \(T_{ox}\ L\ W\) by \(\alpha\)

\textbf{Interconnect:} Connection between different metal layer %TODO replace it correctly and verify it

\textbf{Interconnect scaling:} Dividing \(H_{met}\ H_{met} \ \) Pitch by \(\alpha\)

\begin{table}[!ht]
  \centering
  \begin{tabular}{|l|c|c|}
    \hline
    &Constant&Constant\\
    &Voltage&field\\
    \hline
    \hline
    Dimensions: \(W,\ L,\ T_{ox}\)& \(1/\alpha\) & \(1/\alpha\)\\
    \hline
    Voltages: \(V_{dd},\ V_t\)&1 & \(1/\alpha\)\\
    \hline
    Current par device: \(I_{on}\)&\(\alpha\) & \(1/\alpha\)\\
    \hline
    Capacitance per device: \(C_{gg}\)& \(1/\alpha\) & \(1/\alpha\)\\
    \hline
    Area&\(1/\alpha^2\) & \(1/\alpha^2\)\\
    \hline
    Delay, \(1/f_{clk}\)& \(1/\alpha^2 \) & \(1/\alpha\)\\
    \hline
    Energy per operation& \(1/\alpha\) & \(1/\alpha^3\)\\
    \hline
    Power density&\(\alpha^3\) & 1\\
    \hline
  \end{tabular}
  \caption{Transistors scaling}
  \label{Transitors-scaling}
\end{table}

Limit to constant-voltage scaling: 5V and power
\begin{itemize}
  \item Area overhead and speed penalty
  \item High power density
  \item Total power: battery lifetime concern
\end{itemize}
\bigbreak

\subsubsection{Interconnect limitation}
Limit to constant field scaling: the resistance of interconnect per length unit increases. Reduce the interconnect delay:
\begin{itemize}
  \item New inteconnect (technology)
    \subitem Change material (lower resistance)
    \subitem Change dieletric (void)
  \item New technique (Design)
    \subitem Keep high metal level for global interconnect and not scaled
    \subitem Extensive use of repeater (buffering)
\end{itemize}

\subsubsection{Leakage limitation}
When the gate length becomes too small, the leakage power dominate. To limit the static power dissipation:
\begin{itemize}
  \item Multi \(V_t\) techniques (Design) but high complexity
  \subitem Use low \(V_t\) on critical path
  \subitem Use high \(V_t\) on non-critical path (leakage reduction)
  \item Stop \(V_t\) scaling (Technology) but low speed
    \subitem Limitation \(V_t\) and \(V_{dd}\) (Velocity saturationd and mobility reduction)
    \subitem Strained-Silicon has Enhanced Mobility
\end{itemize}

\subsubsection{Gate leakage current}
The gate leakage exponentially increases with \(T_{ox}\) scaling. Short-channel effects
  \begin{itemize}
  \item Degradation of the slope,
  \item Shift of \(V_t\) (roll-off)
  \item S-D leakage current
  \end{itemize}

  \(\Rightarrow\) Process diversification
  \begin{itemize}
    \item Low standby power (LSTP) CMOS
    \item Low operation power (LOP) CMOS
    \item High performance (HP) CMOS
  \end{itemize}


  \(\Rightarrow\) Increase \(C_{ox}\) to limit short-channel effects \(\Rightarrow\) Change oxide material


\subsubsection{Variability and scalability}
\textbf{Speed binning:} Regroup IC by Normalized leakage and frequency (due to variability - Random dopant fluctuations \textbf{RDF}).

variability on \(V_t\) limit maximum speed and increase total leakage on large chip. It also reduce the SNM for SRAM (failures). To mitigate the variability:
\begin{itemize}
  \item New devices (technology)
    \subitem Finfet: 3D CMOS - Pricer wafer but simpler process (No significant increase)
    \subitem Full-depleted (FD) SOI: thin undoped channel (no RDF) in a buried oxide
  \item New techniques (Design)
\end{itemize}


\subsubsection{Lithography}
Since 180 nm generation, we use sub wavelength lithography
\begin{itemize}
  \item Rounding effects
  \subitem Gate length variation induces high leakage
  \subitem Higher capacitance
  \subitem Short circuit
\item Line edge roughness (\textbf{LER})
  \subitem Varialbe line resistance/capacitance
  \subitem Further \(V_t\) fluctuations
\end{itemize}

Solutions
\begin{itemize}
 \item Optical proximity correction (OPC) -> interference
 \item Resolution enhancement techniques (\textbf{RETs})
      \subitem From 45/40nm: highe numerical aperture and phase shift masks
      \subitem From 32/28nm: Immersion lithography
      \subitem From 22 nm: multiple patterning
      \subitem From 5 nm: Extreme Ultra-Violet (\textbf{EUV}) with 13nm wavelength
\end{itemize}
\(\Rightarrow\) Wafer cost is exploding

