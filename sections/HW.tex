
\subsection{A1}

\subsubsection{Code}

\begin{itemize}
  \item Code: pure code
  \item RO: Read Only (eg. image input)
  \item RW: Read and write (data)
  \item ZI: zero init data (stack and heap)
\end{itemize}

\subsubsection{Memory}
the iROM is a non volatile memory and iRAM is a volatile memory (for stack + heap + \dots). We need ram, because flash memory (ROM) has a limited number of writing and RAM is also low power.

\subsubsection{Encoding error for 64*128}
The heap is built in a ascending way and the stack is built in a descending way. If we go out of boundary we will have corruption
\end{document}


\subsubsection{A2}

The compression rate is data dependent. 

Improvement of cortex M4:
\begin{itemize}
  \item Thumb I and II
  \item Floating point hardware
  \item Harvard architecture
\end{itemize}

Avoid int64 (2 registers) and float (no hardware). Instead of dividing or multiplying try to use shift.


\subsection{A2}

We work we 32-bit register so we should avoi int64 (2 registers). We have no floating point hardware so avoid using float.


\subsection{A3}

The pcb delay module here is to model in the simulation the pcb

%TODO complete

\subsection{A4}
Robust HDL coding.

%TODO complete

\subsection{A5}
A path is between a startpoint and a endPoint. Both for the launch path and the capture path. Relevant ? \(\Rightarrow\) Synchronous or not.

We need to clean signal from I/O (ex: reset). and the clock must have a proper SDC declaration. For GPOUT, we should forward the clock (Hardware or software (ex: using Gpout[9] ))


\subsection{A6}

The tool is under-estimating the activity factor of the different net and of the memories. Golden rule: keep 10\% of each phase (DCMI,encoding,SPI). Clock gating reduce the power but add the gate on the capture/launch path so it can reduce  or increase the slack.



\subsection{A7}
% TODO
