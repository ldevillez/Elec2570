\subsection{A1}

We can find those name at output of the compiler:
\begin{itemize}
  \item Code: pure code
  \item RO: Read Only (eg. image input)
  \item RW: Read and write (data)
  \item ZI: zero init data (stack and heap)
\end{itemize}
\bigbreak
The iROM is a non volatile memory and iRAM is a volatile memory (for stack + heap + \dots). We need ram, because flash memory (ROM) has a limited number of writing and RAM is also low power.

The heap is built in a ascending way and the stack is built in a descending way. If we go out of boundary we will have corruption (for example with the 64*128 image)

\bigbreak
The main is not the first thing executed in a program. We need initialization whchi will perpare the system to run the main. It can be observed after a reset


\subsection{A2}

The compression rate is data dependent. 

Improvement of cortex M4:
\begin{itemize}
  \item Thumb I and II
  \item Floating point hardware
  \item Harvard architecture
\end{itemize}

Avoid int64 (2 registers) and float (no hardware). Instead of dividing or multiplying try to use shift. Also try to simplify mathematical expressions.


\subsection{A3}
There is 4 phase of simulations
\begin{itemize}
  \item JTAG: Programmation of the SoC
  \item DCMI: Fetch of the data from the sensor
  \item Compression: using the JPEG-xs algorithm
  \item SPI: sending the compressed image
\end{itemize}

The \texttt{pcb\_module}  module here is to model in the simulation the pcb (physical connection - we have some capacitance ~\(pF\) to charge ~\(ns\)).

The \texttt{ahb\_slave\_mux} selects the slave to be accessed by the master and multiplexes the \textit{HRDATA} signals of the different slaves. In the project it also select the current master of the AHB bus (Cortex-M0/JTAG) based on TMS signal.


\subsection{A4}
See \textsc{Section} \ref{Robust-HDL-coding}  about Robust HDL coding.


\subsection{A5}
A path is between a startpoint and a endPoint. Both for the launch path and the capture path. Relevant ? \(\Rightarrow\) Synchronous or not.

We need to clean signal from I/O (e.g. reset) so we need to latch the signal. The clock must have a proper SDC declaration (to allow metastability) but it's also true for the I/O signals (e.g. reset). For GPOUT, we should forward the clock (Hardware or software (ex: using Gpout[9] ))


\subsection{A6}

The tool is under-estimating the activity factor (internal cell power) of the different net and of the memories. Golden rule: keep 10\% of each phase (DCMI,encoding,SPI). 

Clock gating reduce the power but add the gate on the capture/launch path so it can reduce  or increase the slack.

\subsection{A7}
We have 3 choice to reduce memory power usage
\begin{itemize}
  \item Use smaller memory (we have oversized memory)
  \item Use small memory to build a bigger one and disabled those memory when we are not using it (memory banking)
  \item Create a low power cache to reduce the number of access to PMEM
\end{itemize}

\subsection{A8}

\begin{align}
  \Delta t &\simeq \cfrac{V_{dd} * C_l}{\frac{1}{2}*\mu*C_{ox} * \frac{W}{L}\left(V_{dd} - V_{th}\right)^2}\\
 P_{SW} &\simeq C_L * V_{dd}^2 * f_{CLK} * \alpha_F * N_{nodes}\\
 P_{leak}&\simeq V_{dd} * \frac{W}{L} * \mu * C_{DEP} * U_{th}^2 * 10^{\frac{-V_{th}}{S}}
\end{align}

The process will influence
\begin{itemize}
  \item \(C_l\)
  \item \(C_{ox}\)
  \item \(W\)
  \item \(L\)
  \item \(V_{th}\)
\end{itemize}

The temperature will influence
\begin{itemize}
  \item \(\mu\)
  \item \(V_{th}\)
\end{itemize}


\bigbreak
\(T_{delay}\), \(T_{setup}\) and \(T_{hold}\) are affected by PVT corner.

\bigbreak
To reach the timing closure
\begin{itemize}
  \item Decrease row utilization
  \item Decrease the frequency
  \item Define false paths
  \item Run multiple optimization
  \item Overconstraint the logic synthesis
\end{itemize}


\subsection{A9}

\textbf{Latency:} Delay between the source and the sink.

\textbf{Skew:} Difference of delay between two sink.

\textbf{Clock uncertainty:} Model the skew compared to an ideal clock. The tool will take this data into account when doing the optimization.

During the the \textit{place} stage, we have a better estimation of the capacitance (physical distance between cells is known) and the power estimation will increase.
\subsection{A10}
Starting point
\begin{itemize}
  \item Throughput limited by the power budget
  \item Mainly dynamic power consumption
\end{itemize}
\bigbreak
Possible choice:
\begin{itemize}
  \item Lower \(V_{dd}\)
    \subitem Lower \(P_{dyn}\)
    \subitem Longer delay
    \subitem (Higher static power)
  \item Lower \(V_t\)
    \subitem Shorter delay
    \subitem (Higher static power)
\end{itemize}

\subsection{A11}
Execution profile
\begin{itemize}
  \item Bit packing: 30\%
  \item Image pre-processing and DWT: 36\%
  \item Data packing: 16\%
\end{itemize}

Slave HW accelerator
\begin{itemize}
  \item data is in the memory
  \item Cortex-M0 must read/write it to the accelerato buffers
  \item SW loop for read/write
  \item +33.8\% Throughput
  \item +13.3\% Power
\end{itemize}


Slave and Master HW accelerator
\begin{itemize}
  \item Data is in the memory
  \item HW accelerator can read/write directly the memory
  \item +46.8\% Throughput
  \item +0.5\% Power
\end{itemize}
% TODO


\bigbreak

HW accelerator with DCMI
\begin{itemize}
  \item Data is already in the peripheral
  \item No accesses overhead
  \item Data needs to be stored
  \item +48.8\% Throughput
  \item +0.8\% Power
  \item +8.6\% Area (Combinational logic +17\%)
\end{itemize}


\subsection{A12}

%TODO: cry
